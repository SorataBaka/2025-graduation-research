\documentclass{jabstract}
\usepackage{url}
\usepackage{titlesec}

\titlespacing*{\section}
{0pt}{1.5ex plus 1ex minus .2ex}{0.9ex plus .2ex}

\titlespacing*{\subsection}
{0pt}{1.25ex plus 1ex minus .2ex}{0ex plus .2ex}

\titlespacing*{\subsubsection}
{0pt}{1ex plus 0.5ex minus .2ex}{0.4ex plus .2ex}

\graphicspath{{images/eps}}
\jtitle{インドネシアにおける2025年軍事法案を巡る世論に関するX(旧Twitter)上の感情分析}
\jauthor{クリスティアン ハルジュノ}
\jaffiliation{情報工学分野}
\jteacher{本間宏利}

\begin{document}
\maketitle

\begin{multicols}{2}
\section{はじめに}
インドネシア国軍法案(RUU TNI)の改正は2025年3月20日に成立し, 第47条をめぐって大きな論争を招いた. この条文は現役軍人の文民職への任命を認めており, 1998年以降の民主化改革を逆行させるとして批判を受けた. これにより, 軍の二重機能(Dwifungsi ABRI)の復活が懸念されている\cite{HRW2024}. 
本研究は, X(旧Twitter)上での議論に焦点を当て, \#TolakRUUTNIなどの抗議的ハッシュタグを通じた国民感情を分析する. 具体的には, 法案提案から成立後までの期間における感情の極性分布を測定し, 不満の主要因を特定する. 分析には, Xの検索エンジンから収集した約20万件のツイートを用いる. 
\section{先行研究}
ソーシャルメディアデータの感情分析, 特にインドネシアにおける本法案問題は, さまざまな手法と一定の成果をもって研究されてきた. Ilhamら(2025)はOrange Data Miningを用いて400件のツイートを分析し, 41.5\%以上がネガティブ感情を示した\cite{Ilham2025}. 一方, Adwinら(2025)は2025年3月1日~31日の投稿をWebスクレイピングで収集し, SVMによる感情分類を実施, 5分割交差検証で平均精度78.99\%, F1スコア83\%を得た\cite{Nurhasananda_Akbar_2025}. これら先行研究に共通する課題は, データの質と量の不足によるモデル精度の低下である. 
\section{研究方法}
本研究では, 特定語句を検索するためにXの検索エンジンを利用する自作Webクローラを用いる~\cite{twitter_search_operators}. クローラは新規投稿を自動取得・解析し, データベースに保存する. ノイズ除去と関連データ抽出には\texttt{indobertweet}~\cite{koto2021indobertweet} を微調整したモデルを用い, さらにTF-IDFによる重要語抽出を行い, 再検索に活用する. これにより, より多くの関連データを効率的に収集できる. 
十分なデータ収集後, 3値の感情スコア(-1, 0, 1)を用いた感情分析モデルを学習し, 時間区間ごとのスコアリングを実施する. 本研究の全体的な処理手順を図~\ref{fig:general-overview} に示す. 

\begin{figurehere}
  \centering
    \includegraphics[width=0.9\linewidth]{research_overview.eps}
    \caption{研究方法の概念}\label{fig:general-overview}
\end{figurehere}

\section{行った事}
本研究では, Puppeteer\footnote{\url{https://github.com/puppeteer/puppeteer}}を用いてNode.js上にWebクローラ\footnote{\url{https://github.com/SorataBaka/2025-graduation-research/tree/main/twitter-parse-v2}}を実装し, 関連・非関連を含む約20万件のサンプルを取得した. ノイズの多いデータに対する性能向上を目的に, サンプリング前の意味的クラスタリング手法を検討した. この副研究は2025年北海道ALU\footnote{\url{https://sites.google.com/view/hokkaidonlp/lau}}年次大会で発表され, MDPI特別号に掲載予定である. 関連性分類モデルの初期学習は, 自身でラベル付けした1000件のサンプルを用いて実施し, その結果を表\ref{tab:initial-model-performance}に示す. 
\begin{tablehere}
  \noindent
  \parbox{\linewidth}{
    \centering
    \caption{ここにキャプションを挿入}\label{tab:initial-model-performance}
    \resizebox{\linewidth}{!}{
      \begin{tabular}{|l|c|c|c|}
        \hline
        クラス & 適合率 (Precision) & 再現率 (Recall) & F1スコア \\
        \hline
        マクロ平均 & 0.9150 & 0.9111 & 0.9129 \\
        \hline
        加重平均 & 0.9180 & 0.9182 & 0.9180 \\
        \hline
      \end{tabular}
    }
  }%
\end{tablehere}
\section{今後の課題}
\begin{enumerate}
\item 関連サンプルを抽出し, TF-IDFを用いて重要なキーワードを取得する. 得られた重要キーワードを利用することで, Xの検索エンジンを再利用し, より高品質で高シグナルなデータを収集して分析に用いることができる.
\item 感情分類モデルを学習させる.
\end{enumerate}
{\small
\bibliographystyle{plain}
\bibliography{references.bib}
}



\end{multicols}

\end{document}
