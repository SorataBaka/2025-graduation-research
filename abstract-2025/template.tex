\documentclass{jabstract}

\jtitle{インドネシアにおける2025年軍事法案を巡る世論に関するX(旧Twitter)上の感情分析}
\jauthor{クリスティアン ハルジュノ}
\jaffiliation{情報工学分野}
\jteacher{本間宏利}

\begin{document}
\maketitle

\begin{multicols}{2}
  
\section{はじめに}
インドネシア国軍法案 (別名 Rancangan Undang-Undang Tentara Nasional Indonesia または RUU TNI)の改正は, 2025年3月20日に法として成立した際, 大きな論争を巻き起こした. 主な論点は特に第47条にあり, この条文は (武器を携帯する権利を持つ)現役軍人が非防衛分野の文民職に任命されることを規定しており, これは1998年以降の民主化改革を覆すものである. これは, 軍の二重機能 (Dwifungsi ABRIとしても知られる)\cite{HRW2024}の復活に関する重大な懸念を引き起こす. 本研究はインドネシアのソーシャルメディアのパラダイム, より具体的には, インドネシア全国の市民にとって公平な議論の場として機能したX (旧Twitter)に焦点を当てる. \#TolakRUUTNIなどのハッシュタグが, インドネシア政府への抗議の一形態として大規模に使用された \cite{CNN2024}.

本研究は, X上での公開された言説の感情分析を行うことにより, RUU TNI 2025に対する国民感情を定量化し分析することを目的とする. 具体的な目的は, 法案が最初に提案された時から正式に署名された後の数ヶ月間にわたる感情の極性の分布を測定し, 国民の不満の主な要因を特定することである. この分析は, Xの検索エンジンから取得した20万件以上のツイートからなるコーパスを使用する.
\section{先行研究}
The sentiment analysis of social media data, particularly regarding this specific legislative issue in Indoensia has been extensively studied with degrees of success and different methodical approaches. One study by Ilham et al (2025) were performed against 400 tweets obtained with orange data mining operation which resulted in mostly negative sentiment of over 41.5\%\cite{Ilham2025}. Another study by Adwin et al (2025) performed sentiment analysis on a time series data between 1st and the 31st of March 2025 scraped from X using the web scraping technique. The sentiment analysis itself was performed using Support Vector Machine (SVM) and evaluated using the 5-Fold Cross Validation method which resulted in an average accuracy of 78.99\% and F1-Score of 83\%\cite{Nurhasananda_Akbar_2025} with 395 samples.
Throughout previous researches found online, a common problem that was experienced by researches is the lack of quality and quantity which severely affected the accuracy of the used models or algorithms. This ``quality'' challenge is particularly pronounced in Indonesian social media data. The text retrieved is often extremely dirty which is characterized by informal slang, non-standard abbreviations, and most significantly, pervasive code-switching between Indoensian, English, and a local dialect (Javanese, Sundanese, Bataknese, etc). These complex phenomena presents a significant hurdle for standard NLP models.

In this study, we will be implementing data mining technique in a limited environment such as X by utilizing TF-IDF to take full advantage of X's search engine as well as data sampling methods that enables us to train a model that generalizes better over noisy and low quality data.

\section{研究方法}
\subsection{データ収集}
\subsection{データ全処理}
\subsection{分析手法}
\section{期待される結果と貢献}
\section{行った事}
\section{結論と今後の課題}

{\small
\bibliographystyle{plain}
\bibliography{references.bib}
}

\end{multicols}

\end{document}
