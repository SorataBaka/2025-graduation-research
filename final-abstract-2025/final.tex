\documentclass{jabstract}
\usepackage{url}
\graphicspath{{images/eps}}
\jtitle{インドネシアにおける2025年軍事法案を巡る世論に関するX(旧Twitter)上の感情分析}
\jauthor{クリスティアン ハルジュノ}
\jaffiliation{情報工学分野}
\jteacher{本間宏利}

\begin{document}
\maketitle

\begin{multicols}{2}
  
\section{背景と目的}
2025年3月のインドネシア国軍法案(RUU TNI)成立は、現役軍人の文民職就任(第47条)による「軍の二重機能」復活の懸念から、広範な社会的論争を招いた\cite{HRW2024}。本研究は、X(旧Twitter)における\#TolakRUUTNI等の抗議投\~25万件を対象とし、法案成立前後の感情極性の推移と不満の主要因を定量的に明らかにする\cite{CNN2024}。
\section{研究方法}

\subsection{生成AIファインチューニングのためのラベリング} 各タスクに対して20,000サンプルのサブセットを用意し, 問題の複雑さに応じてラベリング戦略を適応させた. 二値分類である関連性 (Relevancy) タスクでは, 厳密な論理ゲートを用いたQwen2:7Bを使用した. 一方, 感情分析 (Sentiment) タスクの言語的な曖昧さに対処するため, 文脈を認識するペルソナを持つQwen2.5:7B-Instructを採用し, インドネシア語の微妙な感情のニュアンスを区別するために談話の手がかり (discourse cues) を優先した\cite{yang_qwen2_2024}.

\subsection{クラスタリングにおけるラベル修正} 過度なクリーニング (over-cleaning) による悪影響を避けるため, 学習には未修正のラベルを使用した. 評価用として, Sentence BERT埋め込み上の\textit{Cluster Homogeneity Assumption} (クラスタ均質性の仮定) を用いて\cite{rigollet_generalization_2006}, 15,000サンプルのゴールデンスタンダードを効率的に構築した. 整合性の取れないクラスタを対象に手動レビューを行うことで, ラベルエラーの約2\%を特定し修正した.

\subsection{モデル訓練} 反復的なデータ収集の結果, 重度のネガティブ感情バイアスが生じた. これに対処するため, 関連性モデルには\textit{Class-Weighted Cross-Entropy Loss}を, 感情モデルには\textit{Focal Loss}を適用した. 後者は, 多数派である Negative'' クラスへの過学習を防ぎ, 判別が困難な Neutral'' サンプルの分類性能を大幅に向上させた.
\section{実験結果}
\begin{tablehere}
\noindent
\parbox{\linewidth}{
    \centering
    \caption{関連性分離モデル評価結果}\label{tab:relevancy_metrics}
    \resizebox{\linewidth}{!}{
    \begin{tabular}{|l|c|c|c|}
        \hline
        クラス & 精度率 (Precision) & 再現率 (Recall) & F1スコア \\
        \hline
        無関係 (Irrelevant) & 96.13\% & 90.38\% & 93.17\% \\
        \hline
        関連有 (Relevant) & 92.93\% & 97.21\% & 95.02\% \\
        \hline
        マクロ平均 & 94.53\% & 93.79\% & 94.09\% \\
        \hline
        加重平均 & 94.32\% & 94.24\% & 94.22\% \\
        \hline
    \end{tabular}
    }
}%
\end{tablehere}
\begin{tablehere}
\noindent
\parbox{\linewidth}{
    \centering
    \caption{感情分析モデル評価結果}\label{tab:sentiment_metrics}
    \resizebox{\linewidth}{!}{
    \begin{tabular}{|l|c|c|c|}
        \hline
        クラス & 精度率 (Precision) & 再現率 (Recall) & F1スコア \\
        \hline
        負 (Negative) & 93.98\% & 85.80\% & 89.70\% \\
        \hline
        中立 (Neutral) & 63.03\% & 80.43\% & 70.67\% \\
        \hline
        正 (Positive) & 78.35\% & 82.14\% & 80.20\% \\
        \hline
        マクロ平均 & 78.45\% & 82.79\% & 80.19\% \\
        \hline
        加重平均 & 86.24\% & 84.38\% & 84.95\% \\
        \hline
    \end{tabular}
    }
}%

\begin{figurehere}
    \centering
      \includegraphics[width=\linewidth]{final_graph.eps}
      \caption{研究方法の概念}\label{fig:general-overview}
  \end{figurehere}
\end{tablehere}
{\small
\bibliographystyle{plain}
\bibliography{references.bib}
}
\end{multicols}
\end{document}
